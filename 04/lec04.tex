\documentclass[utf8, russian]{beamer}

\usepackage[utf8]{inputenc}
\usepackage[russian]{babel}
\usepackage{hyperref}
\usepackage{graphicx}
\usepackage{listings}
\usepackage{ucs}

\lstset{
    extendedchars=\true,
    inputencoding=utf8x
}

\usetheme{Warsaw}
\usecolortheme{lily}
\useoutertheme[subsection=false]{smoothbars}
\useinnertheme{circles}
\setbeamertemplate{footline}[page number]{}
\setbeamertemplate{navigation symbols}{}

\renewcommand{\figurename}{} 

\title{Архитектура ЭВМ}
\subtitle{Лекция 4. Чем удобны ЯП высокого уровня?}
\author{к.ф.-м.н. Филонов Павел Владимирович \\ filonovpv@gmail.com}
\date{22 сентября 2013 г.}


\institute[МГТУ ГА] 
{
    Московский Государственный Технический Университет \\
    Гражданской Авиации
}
\begin{document}
    \frame{\titlepage}
    \begin{frame}{Смотрите в этой серии}
        \begin{enumerate}
        \pause
        \item Типы данных и форматы констант
        \pause
        \item Арифметика на x86
        \pause
        \item Disassemble\&Debugging 
        \pause
        \item Jump'ов разных и побольше!
        \pause
        \item Побитовые операции
        \end{enumerate}
    \end{frame}
    \section{Ассемблер NASM}
    \subsection{}
    \begin{frame}[fragile]
        \frametitle{Структура программы}\small
        \begin{verbatim}
; NASM example program
; by Filonov Pavel (filonovpv at gmail.com)
; 23 Jul 2013

%include "stud_io.inc"

global _start

section .data ; секция инициализированных данных
    ; ...

section .bss  ; секция неинициализированных данных 
    ; ...

section .text ; секция исполняемого кода
_start:
    ; ...
    FINISH
        \end{verbatim}
\end{frame}
    \begin{frame}[fragile]
        \frametitle{DB и его друзья}\small
        \begin{verbatim}
;<метка> <псевдоинструкция> <список значений> 
ten dd 10 
msg db "Hello, world"\end{verbatim}
        \begin{block}{Псевдоинструкции в .data}\small
            \begin{verbatim}
db  0x55                 ; просто байт 0x55
db  0x55, 0x56, 0x57     ; три байта по порядку
db  'a', 0x55            ; можно использовать символы
db  'hello', 13, 10, '$' ; как и строки
dw  0x1234               ; 0x34 0x12
dw  'a'                  ; 0x61 0x00 (просто число)
dw  'ab'                 ; 0x61 0x62 (символьная константа)
dw  'abc'                ; 0x61 0x62 0x63 0x00 (строка)
dd  0x12345678           ; 0x78 0x56 0x34 0x12
dd  1.234567e20          ; число с плавающей точкой
dq  0x123456789abcdef0   ; 8-ми байтное число
dq  1.234567e20          ; число двойной точности
dt  1.234567e20          ; число увеличенной точности
            \end{verbatim}
        \end{block}
\end{frame}
    \begin{frame}[fragile]
        \frametitle{RESB и его друзья}\small
        \begin{verbatim}
;<метка> <псевдоинструкция> <размер> 
ten dd 10 
msg db "Hello, world"\end{verbatim}
        \begin{block}{Псевдоинструкции в .bss}
            \begin{verbatim}
buffer      resb    64  ; резервирует 64 байта
wordvar     resw    1   ; резервирует одно слово (2 байта)
regval      resd    1   ; размер одного регистра (4 байта)
realarray   resq    10  ; массив из 10 вещественных чисел\end{verbatim}
        \end{block}
        \begin{block}{EQU}
        \begin{verbatim}
msg db "Hello, world", 0x0a
len equ $-msg\end{verbatim}
        \end{block}
        \begin{block}{INCBIN}
        \begin{verbatim}
incbin "file.dat"          ; вставить весь файл
incbin "file.dat",1024,512 ; пропустить первые 1024 байта и
                           ; и вставить не больше 512 байт\end{verbatim}
        \end{block}
\end{frame}
    \begin{frame}[fragile]
        \frametitle{Числовые константы}
        \begin{verbatim}
200         ; десятичная
0200        ; всё ешё десятичная
0200d       ; явно десятичная
0d200       ; также десятичная
0c8h        ; шестнадцатиричная (hex)
$0c8        ; снова hex (0 обязательно)
0xc8        ; опять hex
0hc8        ; всё ещё hex
310q        ; восьмиричная
0o310       ; опять восьмиричная
0q310       ; и снова восьмиричная
11001000b   ; двоичная
1100_1000b  ; таже двоичная конста
1100_1000y  ; и снова двоичная
0b1100_1000 ; она же
0y1100_1000 ; всё ещё двоичная       
        \end{verbatim}
\end{frame}

    \section{Арифметика}
    \subsection{}
    \begin{frame}[fragile]
        \frametitle{Команда MOV}
        {\bf MOV} (Move) --- набор команд, предназначенных для пересылки данных из одного места в другое
        \begin{block}{Виды операндов}\small
        \begin{verbatim}
mov eax, ebx       ; из одного регистра в другой
mov eax, 112       ; задать(непосредственным операндом)
                   ; значение регистра
mov eax, [count]   ; из памяти в регистр, здесь
                   ; count - адрес в памяти (метка), а
                   ; [count] - значение по этому адресу
mov [count], eax   ; из регистра в память
mov [x], dword 25  ; задать(непосредственным операндом)
                   ; значение ячейки памяти\end{verbatim}
        \end{block}
        \begin{block}{Сцецификаторы размеров операнда}\footnotesize
        \begin{itemize}
            \item byte --- байт
            \item word --- слово (2 байта)
            \item dword --- двойное слово (4 байта)
        \end{itemize}
        \end{block}
\end{frame}
    \begin{frame}[fragile]
        \frametitle{Исполняемый адрес}
        \begin{figure}
        \includegraphics[width=0.7\linewidth]{fig/effective_addr.pdf}
        \end{figure}
        {\bf LEA} (Load Effective Address) --- загружает в регистр вычисленное значение исполняемого адреса. Обращение к памяти не производится
        \begin{verbatim}
lea edi, [array + eax + 4*ebx]
        \end{verbatim}
\end{frame}
    \begin{frame}{Пример}
    \end{frame}
    \begin{frame}[fragile]
        \frametitle{Команды сложения и вычитания}
        {\bf ADD} (Addition) --- целочисленное сложение

        {\bf SUB} (Substraction) --- целочисленное вычитание
        \begin{block}{Примеры}
            \begin{verbatim}
add eax, ebx        ; eax := eax + ebx
add al,  cl         ; al := al + cl
add al,  10         ; al := al + 10
sub ebx, [a]        ; ebx := ebx - [a] 
sub [a], dword 100  ; [a] := [a] - 100
            \end{verbatim}
        \end{block}
        Команды ADD и SUB устанавливают флаги (ZF, SF, OF, CF, PF)

        {\bf ADC} (Add with Carry) --- сложить с переносом (к сумме прибавляется значение флага CF)

        {\bf SBB} (Substraction with Borrow) --- вычитание с заимствованием (из разности вычитается CF)
\end{frame}
    \begin{frame}[fragile]
        \frametitle{Команды \tt inc, dec, neg и cmp}
        {\bf INC} (Increase) --- увеличивает значение операнда на единицу

        {\bf DEC} (Decrease) --- уменьшает значение операнда на единицу

        Команды INC, DEC устанавливают флаги ZF, OF, SF (но не CF)

        \bigskip
        {\bf NEG} (Negative) --- изменяет знак числа (вычисляет дополнительный код)

        {\bf CMP} (Compare) --- вычитает второй операнд из первого (результат не сохраняется). CMP используется ради установки флагов и обычно за ним следует условный переход
\end{frame}
    \begin{frame}[fragile]
        \frametitle{Беззнаковое умножение и деление}
        {\bf MUL } (multiplication) --- беззнаковое целочисленное умножения (в качестве операнда указывается только один множитель, второй задан неявно разрядностью операции)

        {\bf DIV } (division) --- безнаковое челочисленное деление (в качестве операнда указывается делитель, делимое задётся неявно разрядностью операции)

        {\bf При выполнении команды DIV не забудьте обнулить старшую часть делимого!}


        \begin{table}\small
            \begin{tabular}{|p{1cm}|p{2cm}|p{2cm}||c|c|c|}
                \hline Разря-дность & \multicolumn{2}{|c||}{Умножение} & \multicolumn{3}{|c|}{Деление} \\
                \hline  & неявный \newline множитель & результат \newline умножения & делимое & частное & остаток \\
                \hline 8 & AL & AX &  AX & AL & AH \\
                \hline 16 & AX & DX:AX & DX:AX & AX & DX \\
                \hline 32 & EAX & EDX:EAX & EDX:EAX & EAX & EDX \\
                \hline
            \end{tabular}
        \end{table}
\end{frame}
    \begin{frame}[fragile]
        \frametitle{Знаковое умножение и деление}
        {\bf iMUL } (multiplication) --- знаковое целочисленное умножение. Имеет форму с одним, двумя или тремя операндами.

        {\bf iDIV } (division) --- знаковое целочисленное деление (в качестве операнда указывается делитель, делимое задётся неявно разрядностью операции)

        {\bf CBW, CWD, CWDE, CDQ} --- увеличение разрядности знакового числа. cbw расширяет AL до AX, cwd AX до DX:AX, cwde AX до EAX, cdq EAX до EDX:EAX
        \begin{table}\small
            \begin{tabular}{|c|c|c|}
                \hline Кол-во операндов & Запись & Описание \\
                \hline 1 & imul r/m & Аналогично mul \\
                \hline 2 & imul r, r/m & Первый := Первый*Второй \\
                \hline 3 & imul r, r/m, imm & Первый := Второй*Третий \\
                \hline
            \end{tabular}
        \end{table}
\end{frame}
    \section{GDB}
    \subsection{}
    \begin{frame}{Отладчик gdb}
    Программа {\tt gdb} предназначение для отладки программ на различных языках программирования.

    {\bf Использование:} gbd программа
    \begin{block}{основные команды}
        \begin{itemize}
            \item run (r) --- начать выполнение программы
            \item break (b) --- установить точку остановки
            \item next instruction (ni) --- следующая инструкция
            \item step instruction (si) --- следующая инструкция (с заходом в подпрограммы)
            \item continue (c) --- продолжить выполнение
            \item info registers (i r) --- показать значение регистров
            \item print (p)--- распечатать значение регистра или памяти
            \item quit (q) --- выход
        \end{itemize}
    \end{block}
    \end{frame}
    \section{Jumps}
    \subsection{}
    \begin{frame}{Условные и безусловные переходы}
        \begin{block}{Виды переходов}
            \begin{itemize}
                \item {\bf Дальние} (far) --- для передачи управления в другой сегмент (не используются в <<плоской>>> модели памяти)
                \item {\bf Близкие} (near) --- для передачи управление в произвольное место внутри одного сегмента
                \item {\bf  Короткие} (short) --- переходы, используемые для оптимизации и позволяющие прыгать не более чем на 127 байт вперёд и не менее чем на 128 байт назад.
             \end{itemize}
        \end{block}
        Безусловные переходы по умолчанию объявляются близкими.

        Условные переходы по умолчанию объявляются короткими. 

        Если адрес перехода на попадает в необходимый диапазон, то nasm выдаст ошибку следующего вида:

        {\tt error: short jump is out of range}
    \end{frame}
    \begin{frame}{Условные и безусловные переходы}
        \begin{block}{Виды переходов}
            \begin{itemize}
                \item {\bf Дальние} (far) --- для передачи управления в другой сегмент (не используются в <<плоской>>> модели памяти)
                \item {\bf Близкие} (near) --- для передачи управление в произвольное место внутри одного сегмента
                \item {\bf  Короткие} (short) --- переходы, используемые для оптимизации и позволяющие прыгать не более чем на 127 байт вперёд и не менее чем на 128 байт назад.
             \end{itemize}
        \end{block}
        Безусловные переходы по умолчанию объявляются близкими.

        Условные переходы по умолчанию объявляются короткими. 

        Если адрес перехода на попадает в необходимый диапазон, то nasm выдаст ошибку следующего вида:

        {\tt error: short jump is out of range}
    \end{frame}
    \begin{frame}{Условные переходы по отдельным флагам}
        \begin{table}
            \begin{tabular}{|c|c||c|c|}
                \hline команда & условие & команда & условие \\
                \hline jz & ZF = 1 & jnz & ZF = 0 \\
                \hline js & SF = 1 & jns & SF = 0 \\
                \hline jc & CF = 1 & jnc & CF = 0 \\
                \hline jo & OF = 1 & jno & OF = 0 \\
                \hline jp & PF = 1 & jnp & PF = 0 \\
                \hline
            \end{tabular}
        \end{table}
    \end{frame}
    \begin{frame}{Условные переходы по результатам сравнений}
        \begin{table}\footnotesize
            \begin{tabular}{|c|c|c|c|}
                \hline команда & jump if & Выражение & Условие \\
                \hline je & equal & $a=b$ & ZF = 1 \\
                       jne & not equal & $a \ne b$ & ZF = 0 \\
                \hline jl & less & $a < b$ & SF $\ne$ OF \\
                       jnge & not greater or equal  & & \\
                \hline jle & less or equal & $a \le b$ & SF $\ne$ OF или ZF = 1\\
                       jng & not greater & & \\
                \hline jg  & greater & $a > b$ & SF=OF и ZF = 0 \\
                       jnle & not less or equal & & \\
                \hline jge & greater or equal & $a \ge b$ & SF=OF \\
                       jnl & not less & & \\
                \hline jb & below & $a < b$ & CF = 1 \\
                       jnae & not above or equal  & & \\
                \hline jbe & below or equal & $a \le b$ & CF=1 или ZF = 1\\
                       jna & not above & & \\
                \hline ja  & above & $a > b$ & CF=0 и ZF = 0 \\
                       jnbe & not less or equal & & \\
                \hline jae & above or equal & $a \ge b$ & CF=0 \\
                       jnb & not below & & \\
                \hline
            \end{tabular}
        \end{table}
    \end{frame}
    \begin{frame}{Условные переходы и ECX}
        Регистр {\bf ECX} традиционно используется как счётчик цикла и в архитектуре x86 присутсвует несколько специальных условных переходов, связанных со значением данного регистра.

        \bigskip
        {\bf loop} --- уменьшает значение {\bf CX} и делает которкий переход, если CX не равен нулю.

        \bigskip
        {\bf jcxz} (jump if CX is zero) --- производит условный переход, если значение счётчика {\bf CX} равно нулю.

        \bigskip
        {\bf loope, jecxz} --- аналогичные команды, работающие со значением регистра {\bf ECX}
    \end{frame}
    \section{Побитовые операции}
    \subsection{}
    \begin{frame}{Побитовые операции}\footnotesize
        \begin{block}{логические операции}
            \begin{itemize}
                \item {\bf AND} --- логическое "И"
                \item {\bf OR} --- логическое "ИЛИ"
                \item {\bf NOT} --- логичское "НЕ"
                \item {\bf XOR} --- исключающее "ИЛИ"
                \item {\bf TEST} --- выполняет AND, но только устанавливает флаги
            \end{itemize}
        \end{block}
        \begin{block}{операции сдвига}
            \begin{itemize}
                \item {\bf SHR} (shift right) --- простой сдвиг вправо (деление на $2^n$)
                \item {\bf SHL} (shift left) --- простой сдвиг влево (умножение на $2^n$)
                \item {\bf SAR} (shift arithmetic right) --- арифметический сдвиг вправо (знаковое деление на $2^n$)
                \item {\bf SAL} (shift arithmetic left) --- арифметический сдвиг влево (синоним {\bf SHL})
                \item {\bf ROR} (rotate right) --- циклический сдвиг вправо
                \item {\bf ROL} (rotate left) --- циклический сдвиг влево
            \end{itemize}
        \end{block}
    \end{frame}
\end{document}