\documentclass[utf8, russian]{beamer}

\usepackage[utf8]{inputenc}
\usepackage[russian]{babel}
\usepackage{hyperref}
\usepackage{graphicx}
\usepackage{listings}
\usepackage{ucs}
\usepackage{clrscode}

\lstset{
    extendedchars=\true,
    inputencoding=utf8x
}

\usetheme{Warsaw}
\usecolortheme{lily}
\useoutertheme[subsection=false]{smoothbars}
\useinnertheme{circles}
\setbeamertemplate{footline}[page number]{}
\setbeamertemplate{navigation symbols}{}

\renewcommand{\figurename}{} 

\title{Архитектура ЭВМ}
\subtitle{Лекция 5. Стек и подпрограммы}
\author{к.ф.-м.н. Филонов Павел Владимирович \\ filonovpv@gmail.com}
\date{22 сентября 2013 г.}


\institute[МГТУ ГА] 
{
    Московский Государственный Технический Университет \\
    Гражданской Авиации
}
\begin{document}
    \frame{\titlepage}
    \begin{frame}{Сегодня поговорим о ...}
        \begin{itemize}
            \pause
            \item Что такое стек?
            \pause
            \item Зачем он нужен?
            \pause 
            \item Чем помогут подпрограммы?
            \pause
            \item А как они вообще работают?
            \pause
            \item Стековый фрейм --- взрыв мозга или неизбежное зло?
            \pause
            \item Разные соглашения о вызове подпрограмм
        \end{itemize}
    \end{frame}
    \section{Стек}
    \subsection{}
    \begin{frame}{Стек}

        {\bf Стек } ({\it англ.} Stack) --- структура данных, организованная по принципу LIFO (Last In --- First Out, последним пришёл --- первым вышел)
    \begin{columns}
        \column{0.5\linewidth}
            \includegraphics[width=\linewidth]{fig/stack_rails.jpg}
        \column{0.5\linewidth}
            \includegraphics[width=\linewidth]{fig/stack.png}
    \end{columns}
    {\bf Push} --- положить элемент на вершину стека

    {\bf Pop} --- снять элемент с вершины стека
    \end{frame}
    \begin{frame}{Организация стека в архитектуре x86}
        \begin{columns}
        \column{0.5\linewidth}
        \begin{itemize}
            \item Секция стека располагается в старших адресах памяти. 
            \item Адрес вершины стека хранится в регистре ESP.
            \item Вершина стека растёт в сторону уменьшения адресов. 
        \end{itemize}
        \column{0.5\linewidth}
            \includegraphics[width=\linewidth]{fig/segments.pdf}
        \end{columns}
    \end{frame}
    \begin{frame}{Команды работы со стеком}
        \begin{block}{Положить в стек}\small
            Команда {\bf push} помещает операнд на вершину стека и уменьшает ESP на размер операнда.
            \begin{itemize}
                \item push --- положить слово (2 байта) или двойное слово (4 байта)
                \item pushf (pushfd) --- положить в стека EFALGS (4 байта)
                \item pusha (pushad) --- положить в стек все 4-байтовые регистры (EAX, ECX, EDX, EBX, ESP, EBP, ESI, EDI) 
            \end{itemize}
        \end{block}
        \begin{block}{Взять со стека}\small
            Команда {\bf pop} извлекает с вершины стека данные, записывает их в операнд и увеличивает ESP на размер операнда
            \begin{itemize}
                \item pop --- извлечь слово (2 байта) или двойное слово (4 байта)
                \item popf (popfd) --- извлечь из стека EFALGS (4 байта)
                \item popa (popad) --- извлечь из стека все 4-байтовые регистры (EDI, ESI, EBP, EBX, EDX, ECX, EAX). Значение ESP пропускается             
            \end{itemize}
        \end{block}
    \end{frame}
    \begin{frame}{Пример (правильное скобочное выражение)}
        \begin{codebox}
            \Procname{Правильное скобочное выражение}
            \li \Comment возвращает True, либо False
            \li c $\leftarrow$ следующий символ
            \li \While не конец файла
            \li \Do \If c = '(' 
            \li     \Then push c
            \li     \Else \If c = ')'
            \li           \Then pop t
            \li                 \If t $\ne$ c
            \li                 \Then \Return False
                                \End
                          \End
                    \End
            \li     c $\leftarrow$ следующий символ
                \End
            \li \If стек не пуст
            \li \Then \Return False
            \li \Else \Return True
        \end{codebox}
    \end{frame}

    \section{Подпрограммы}
    \subsection{}
    \begin{frame}{Подпрограммы}

        {\bf Подпрограммой} называется некоторая обособленная часть программного кода, которая может быть {\bf вызвана} из главной программы (или из другой подпрограммы.)
        \bigskip

        Под {\bf вызовом} понимается временная передача управления подпрограмме с тем, чтобы, когда подпрограмма сделает свою работу, она вернула управление в точку, откуда её вызвали.

        \bigskip
        При вызове подпрограммы необходимо запомнить {\bf адрес возврата}, т.е. адрес машинной команды, следующей за командой вызова подпрограммы.
        \bigskip

        При {\bf вызове} подпрограмме могут передаваться параметры, влияющие на её работу.
        \bigskip

        Подпрограммы используют в ходе своей работы {\bf локальные переменные}.
    \end{frame}
    \begin{frame}{Стековый фрейм}
        \begin{columns}
        \column{0.3\linewidth}
        В современных вычислительных системах {\bf параметры}, {\bf адрес возврата} и {\bf локальные переменные} подпрограмм хранятся в области стековой памяти, называемой {\bf стековым фреймом}.
        \column{0.7\linewidth}
        \includegraphics[width=\linewidth]{fig/stack_frame.pdf}
        \end{columns}
    \end{frame}
    \begin{frame}{Вызов и возврат из подпрограммы}\small
        Инструкция {\tt call <метка>} сохраняет в стеке адрес возврата и передаёт управление на указанную метку.

        Инструкция {\tt ret} выталкивает из стека адрес возврата и передаёт управление на него.

        \begin{block}{Действия при входе в подпрограмму}\small
            \begin{enumerate}
                \item сохранить в стеке значение EBP
                \item записать в EBP текущий адрес вершины стека (ESP)
                \item вычесть из ESP число байт под локальные переменные
                \item сохранить в стеке все регистры, которые будут изменены в подпрограмме (включая EFLAGS)
            \end{enumerate}
        \end{block}

        \begin{block}{Действия при выходе из подпрограммы}\small
            \begin{enumerate}
                \item восстановить значение всех регистров (включая EFLAGS)
                \item записать в ESP значение EBP
                \item восстановить старое значение EBP из стека
                \item вернуться в основную программу ({\tt  ret})
            \end{enumerate}
        \end{block}
    \end{frame}
    \begin{frame}[fragile]
        \frametitle{Пример (Алгоритм Евклида)}
        \begin{columns}[T]
            \scriptsize
            \column{0.4\linewidth}
            Главная программа
            \begin{verbatim}
%include "stud_io.inc"

global _start

section .text
_start:
    push 169    ; b := 169
    push 39     ; a := 39
    call gcd    ; eax := gcd(a,b)
    add esp, 8  ; 
    push eax    
    call printint
    add esp, 4

    FINISH
            \end{verbatim}
            \column{0.6\linewidth}
            Подпрограмма {\tt gcd}
            \begin{verbatim}
gcd:    push ebp
        mov ebp, esp
        push ebx
        push edx
        pushf
while:  cmp [ebp+12], dword 0 ; b <> 0
        je wend
        mov eax, [ebp+8]  ; eax := a
        mov ebx, [ebp+12] ; ebx := b
        cdq               ; eax -> [edx:eax]
        idiv ebx          ; edx := a mod b
        mov [ebp+8],  ebx ; a := b
        mov [ebp+12], edx ; b := edx
        jmp while
wend:   mov eax, [ebp + 8] ; return a
        popf
        pop edx
        pop ebx
        mov esp, ebp
        pop ebp
        ret
            \end{verbatim}
        \end{columns}
\end{frame}
    \begin{frame}[fragile]
        \frametitle{Соглашения о вызове подпрограмм}
    \begin{block}{Что регламентируется:}\small
        \begin{itemize}
            \item как передаются параметры (через регистры или через стек)
            \item в каком порядке
            \item кто восстанавливает стек (вызываемая или вызывающая подпрограмма)
            \item как возвращается результат
        \end{itemize}
    \end{block}
        \centerline{func(fd, msg, len)}
        \begin{columns}
            \column{0.5\linewidth}
            \centerline{cdelc:}
            \begin{columns}
                \column{0.5\linewidth}
                    \begin{figure}
                        \includegraphics[width=\linewidth]{fig/cdecl.pdf}
                    \end{figure}
                \column{0.5\linewidth}
                \begin{verbatim}
push len
push msg
push fd
call func
add esp, 12
                \end{verbatim}
            \end{columns}

            \column{0.5\linewidth}
            \centerline{pascal:}
            \begin{columns}
                \column{0.5\linewidth}
                    \begin{figure}
                        \includegraphics[width=\linewidth]{fig/pascal.pdf}
                    \end{figure}
                \column{0.5\linewidth}
                \begin{verbatim}
push fd
push msg
push len
call func

                \end{verbatim}
            \end{columns}
        \end{columns}
\end{frame}
\end{document}
